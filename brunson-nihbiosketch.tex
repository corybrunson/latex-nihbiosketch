%!TEX TS-program = xelatex
\documentclass{nihbiosketch}

% Journal of Statistical Software
\newcommand{\pkg}[1]{{\fontseries{b}\selectfont #1}}
%\newcommand{\pkg}[1]{{\normalfont\fontseries{b}\selectfont #1}}

\name{Brunson, Jason Cory}
\eracommons{BRUNSONJ}
\position{Research Assistant Professor}

\begin{document}
%------------------------------------------------------------------------------

\begin{education}
Virginia Tech, Blacksburg & B.S. & 08/2000 & 05/2004 & Mathematics \\
Virginia Tech, Blacksburg & B.S. & 08/2001 & 08/2004 & Statistics \\
Virginia Tech, Blacksburg & M.S. & 08/2004 & 05/2005 & Mathematics \\
Virginia Tech, Blacksburg & Ph.D. & 08/2005 & 12/2013 & Mathematics \\
UConn Health, Farmington & Postdoc & 06/2014 & 08/2017 & Quantitative Medicine \\
UConn Health, Farmington & Postdoc & 08/2017 & 04/2020 & Network Modeling in Healthcare \\
\end{education}

\section{Personal Statement}

\begin{statement}

%OVERVIEW
I recently completed two postdoctoral fellowships at UConn Health in the Center for Quantitative Medicine (CQM), following my PhD in Mathematics, a research assistantship in social network analysis, and an adjunct professorship.
I was first hired in 2014 to train as a data scientist under the supervision of Dr. Reinhard Laubenbacher, and I pursued a variety of projects with different collaborators during this time.
In 2017, I was awarded a fellowship with the UConn--NIDCR T90/R90 Research Training Program under Dr. Mina Mina. My project focused on network analysis and modeling of routinely-collected healthcare data sets (``health data'').
I have worked at the University of Florida since 2020 as an Assistant Professor, during which time I have initiated collaborations with several colleagues. I arrived at the focus of this proposal through several strong connections within the Division of Pulmonary, Critical Care, and Sleep Medicine.

%RESEARCH
My research focuses on predictive and exploratory modeling approaches to biomedical problems, in particular using health data. The complexity of these data and of the models trained on them  can conspire to impede understanding and interpretation, and by exploiting tools from topology---the mathematical study of continuity, grounded in measures of distance or similarity---we are able in some settings to improve both the accuracy of predictions and the clinical value of the model components. This work in part responds to my previous efforts to synthesize and evaluate network models in systems medicine, which revealed important limitations of conventional approaches.
Nevertheless, I have maintained an active research program in network science fueled by collaborations with specialists in biological and clinical domains, including cell biology, immunology, and psychology.
Most of my work includes a software development component, which has been important to its reproducibility and accessibility to trainees and collaborators as well as other researchers.
This proposal comprises a specialization in lung transplant outcomes, which is a field rich with open questions and clinical and biological data suitable for new methodological approaches.

%MENTORSHIP
Dr. Laubenbacher also introduced me to research mentorship: I supervised four students on a Research Experience for Undergraduates (REU) during my doctoral program, perhaps my most rewarding experience as a PhD student. I continued the project alongside my dissertation, and since completing my degree I have taken every opportunity to involve students in research. At a summer REU at UConn Health, I mentored two students on several predictive modeling experiments using health data. I also initiated CQM's participation in the High School Mentorship Program, through which I mentored four student interns from two area high schools over three summers and helped connect several more with colleagues. At the University of Florida, I am mentoring medical, graduate, and pre-med undergraduate students on projects related to this proposal. These trainees participated in study design, data wrangling, software development, and computational experiments, and presented their work at symposia and conferences as well as coauthored journal articles, software packages, and software tutorials.

%PROJECT
This background, in particular using network analysis and health data, has prepared me to design and implement rigorous and robust similarity-based models, drawing from artificial intelligence and machine learning as well as classical statistics, as outlined in the proposal.
My experience working with medical and health researchers has enabled me to prepare a training plan to bridge my computational work with the specific biological, clinical, psychosocial, and informatical characteristics and context of lung transplantation.
Finally, my mentorship history has provided valuable experience toward organizing a research group oriented toward the aims of the proposal.

\begin{enumerate}

\item \strong{J.C. Brunson}, X. Wang, and R.C. Laubenbacher (2017). Effects of research complexity and competition on the incidence and growth of coauthorship in biomedicine, \emph{PLoS One} 12(3): e0173444.
\item \strong{J.C. Brunson} and R.C. Laubenbacher (2018). Applications of network analysis to routinely collected healthcare data: a systematic review, \emph{Journal of the American Medical Informatics Association} 25(2): 210--221.
\item \strong{J.C. Brunson}, T.P. Agresta, and R.C. Laubenbacher (2019). Sensitivity of comorbidity network analysis. \emph{JAMIA Open} 3(1): 94--103.
\item M. Terasaki, \strong{J.C. Brunson}, and J. Sardi (2020). Analysis of the three dimensional structure of the kidney glomerulus capillary network, \emph{Scientific Reports} 10, 20334.

\end{enumerate}

\end{statement}

%------------------------------------------------------------------------------
\section{Positions, Scientific Appointments, and Honors}

%\subsection*{Positions}
\begin{datetbl}
2020--     & Research Assistant Professor, Laboratory for Systems Medicine, University of Florida, Gainesville, FL  \\
2017--2020 & Postdoctoral Fellow, Skeletal, Craniofacial \& Oral Biology Training Program, UConn Health, Farmington, CT \\
2014--2017 & Postdoctoral Fellow, Center for Quantitative Medicine, UConn Health, Farmington, CT \\
2014       & Adjunct Professor, Department of Mathematics, Radford University, Radford, VA \\
2010--2013 & Research Assistant, Virginia Bioinformatics Institute, Virginia Tech, Blacksburg, VA \\
\end{datetbl}

%\subsection*{Honors}
%\begin{datetbl}
%2017       & Union Representative of the Year, UHP \\
%\end{datetbl}

%------------------------------------------------------------------------------

\section{Contribution to Science}

\begin{enumerate}

\item \emph{Network science.}
As a Research Assistant and mentor, together with a team of undergraduates, I was introduced to network analysis and scientometrics at a summer program organized around systems biology.
Our study of the mathematics literature, and a later collaboration focused on biomedicine, combined conventional and original tools to describe global changes in collaboration patterns in both communities over recent decades, and our results have been cited in both follow-up research and several commentaries on scientific practice.
I have since kept up an active program of network science, which has involved collaborations with domain experts in immune response and cell biology on original applications of advanced graph theory to modeling cell signaling and capillary development.
Network science has also been a major part of my more recent work involving administrative healthcare data, as detailed in the next section.

\begin{enumerate}
\item \strong{J.C. Brunson}, S. Fassino, A. McInnes, M. Narayan, B. Richardson, C. Frank, P. Ion, and R.C. Laubenbacher (2014). Evolutionary events in a mathematical sciences research collaboration network, \emph{Scientometrics} 99(3): 973--998.
\item \strong{J.C. Brunson}, Triadic analysis of affiliation networks (2015). \emph{Network Science} 3(4): 480--508.
\item \strong{J.C. Brunson}, X. Wang, and R.C. Laubenbacher (2017). Effects of research complexity and competition on the incidence and growth of coauthorship in biomedicine, \emph{PLoS One} 12(3): e0173444.
\item M. Terasaki, \strong{J.C. Brunson}, and J. Sardi (2020). Analysis of the three dimensional structure of the kidney glomerulus capillary network, \emph{Scientific Reports} 10, 20334.
\end{enumerate}

\item \emph{Administrative healthcare data.}
My appointment at the Center for Quantitative Medicine entailed self-training as a ``data scientist''---a combination of statistical literacy, computational programming, and consultancy---with a focus on the modeling and analysis of administrative data sets such as billing claims and electronic health records.
One project in this role was a training collaboration with the Office of the State Comptroller of Connecticut, in which I combined pharmacy and insurance claims data provided by the state with publicly available healthcare survey data to describe how several out-of-state compound pharmacies pushed product to patients at exorbitant costs to the state. This project pushed me to adopt reproducible documentation as a research standard, which also proved useful for updating reports as new data became available, and the visualization software I wrote to communicate some results is used by thousands of users each month.
I have also been engaged in software development with a pharmacist and medical researcher to expedite comparative effectiveness research using claims databases. Hundreds of validated specifications for index events, outcome measures, eligibility criteria, and covariates have been incorporated into a simple framework to extract and process the necessary data from standardized tables, substantially reducing the research overhead.
More recently, I conducted a systematic review of studies that employ network analysis techniques for secondary use of healthcare data. This literature contains myriad projects in diverse domains that in several respects stand to benefit from sharing and combining techniques, and my supervisor and I brought them together into a single reference (and methodological synthesis) we hope will improve collaborations and accelerate advances.
One lesson from this review was that the increasingly popular use of ``comorbidity networks'' suffers from inconsistent methodology and uncertain validity. In response, and with a family physician and health informaticist colleague, we conducted a sensitivity analysis of the comorbidity network construction and several network statistics that have been used to characterize them, which we hope will encourage more consistent and transparent use of this concept in the future.

\begin{enumerate}
\item \strong{J.C. Brunson} and C. Coleman. cerms: Efficient comparative effectiveness research using MarketScan, R package collection, in development. \nolinkurl{https://bitbucket.org/corybrunson/cerms/}
\item \strong{J.C. Brunson} and R.C. Laubenbacher (2018). Applications of network analysis to routinely collected healthcare data: a systematic review, \emph{Journal of the American Medical Informatics Association} 25(2): 210--221.
\item \strong{J.C. Brunson}, T.P. Agresta, and R.C. Laubenbacher (2020). Sensitivity of comorbidity network analysis. \emph{JAMIA Open} 3(1): 94--103.
\item \strong{J.C. Brunson} (2020). ggalluvial: Layered Grammar for Alluvial Plots, \emph{Journal of Open Source Software} 5(49): 2017.
\end{enumerate}

%\item \emph{Topological data analysis.}
%My scientometric work raised a remarkably under-studied question: How can triads of actors be understood in an affiliation network context? A satisfactory answer required the machinery of homological algebra, which had been essential to my dissertation project but which had not been adapted to this setting. As I came to appreciate the limitations of network analysis to exploratory analysis, predictive modeling, and hypothesis testing with healthcare data, this success was an important factor in my decision to pursue topological data analysis (TDA). My TDA work, joint with mathematical and clinical collaborators, is focused on two entangled research programs. One is to couple topological methods with complex predictive models in order to identify health-related features and patient phenotypes that help explain the predictions and the performances of these models. This is joint work with mathematical and clinical collaborators, including MD--PhD student Yara Skaf, who has been instrumental to its design and conduct since she joined our lab and will co-author our first manuscript. The other is to synthesize existing and original TDA tools into a collection of software packages integrated into widely-used data science conventions, in order to make rigorous and efficient topological methods accessible to data scientists. This is joint work with student researchers at several institutions, including Ms. Skaf, and serves an important role in allowing student researchers to participate directly in methods development and experimentation. My own primary project is a topological approach to the problem of the unequal performance of predictive models in healthcare, which connects to work in health disparities, algorithmic bias, and algorithmic stewardship.

%\begin{enumerate}
%\item \strong{J.C. Brunson} and Y. Skaf, Fixed and adaptive landmark sets for finite metric spaces.
%\item R.R. Wadhwa, M. Piekenbrock, and \strong{J.C. Brunson}. tdaverse: An R package collection for topological data analysis. \nolinkurl{http://systemsmedicine.pulmonary.medicine.ufl.edu/software/r-package-ecosystem-tda/}
%\end{enumerate}

%More recently, I mentored a team of undergraduates in developing risk models for myocardial infarction (MI) using the open-access critical care database MIMIC-III. We built analysis pipelines to contribute to the MIMIC-III code repository and built several MI triage and risk models, which are being written up for submission to a medical journal.

\end{enumerate}

\subsubsection*{Complete List of Published Work in MyBibliography:}
\url{https://www.ncbi.nlm.nih.gov/myncbi/browse/collection/47860258/?sort=date&direction=ascending}

%\subsubsection*{Complete List of Publications at ORCiD:}
%\url{https://orcid.org/0000-0003-3126-9494}

%\subsubsection*{Complete List of Reviews at Publons:}
%\url{https://publons.com/researcher/1676577/jason-cory-brunson/peer-review/}


\end{document}
