%!TEX TS-program = xelatex
\documentclass{nihbiosketch}

% Journal of Statistical Software
\newcommand{\pkg}[1]{{\fontseries{b}\selectfont #1}}
%\newcommand{\pkg}[1]{{\normalfont\fontseries{b}\selectfont #1}}

\name{Brunson, Jason Cory}
\eracommons{BRUNSONJ}
\position{Assistant Professor}

\begin{document}
%------------------------------------------------------------------------------

\begin{education}
Virginia Tech, Blacksburg VA & B.S. & 05/2004 & Mathematics \\
Virginia Tech, Blacksburg VA & B.S. & 08/2004 & Statistics \\
Virginia Tech, Blacksburg VA & M.S. & 05/2005 & Mathematics \\
Virginia Tech, Blacksburg VA & Ph.D. & 12/2013 & Mathematics \\
UConn Health, Farmington CT & Postdoc & 08/2017 & Quantitative Medicine \\
UConn Health, Farmington CT & Postdoc & 04/2020 & Network Modeling in Healthcare \\
\end{education}


\section{Personal Statement}

\begin{statement}

%OVERVIEW
Following my PhD in Mathematics, a research assistantship in social network analysis, and an adjunct teaching professorship, I completed two postdoctoral fellowships at UConn Health in the Center for Quantitative Medicine (CQM).
I was first hired in 2014 to train as a data scientist under the supervision of Dr. Reinhard Laubenbacher.
During this time, I investigated the structure and determinants of scientific collaboration and applications of network analysis to routinely-collected healthcare data (``health data'').
In 2017, I was awarded a fellowship with the UConn--NIDCR T90/R90 Research Training Program under Dr. Mina Mina.
My project initially focused on network models of comorbidity, then grew to include topological data analysis, machine learning, and open source software development.
I have worked at the University of Florida since 2020 as an Assistant Professor.

%PROGRAM


%COLLABORATIONS
I collaborate with colleagues in several specialties to \uline{apply descriptive, predictive, and interpretable modeling} to biomedical problems:
I supervised a study of localized models to predict and identify personalized risk factors for readmission and mortality among critical care patients and recently completed a follow-up study on COVID-19 infection.
I have collaborated with a team of psychiatrists and psychologists to identify personal and societal determinants of occupational burnout and behavioral treatment response.
I am continuing a project to build clinical decision support tools from patterns in health data to better stratify patients for alpha-1 antitrypsin screening.
Finally, I worked with a team of specialists in interstitial lung disease and lung transplantation to detect, measure, and explain socioeconomic disparities in post-transplant outcomes, and more recently to identify predictors of severe primary graft dysfunction in lung recipients.
I am currently leveraging my quantitative background and my collaborations in lung transplantation and computational modeling toward biomarker discovery, pathway analysis, and mechanistic modeling of outcomes following lung transplantation.

%PROJECT
\strong{With regard to this proposal,} I will work closely with colleagues on the HIV/AIDS project, collaborate frequently with colleagues in the Model and Data Sharing Core, and liaise frequently between these.
Specifically:
I will work with Dr. Salemi and his colleagues to acquire, assemble, pre-process, tidy, and analyze data sets from the Florida Department of Health and from the OneFlorida+ Data Trust in preparation for modeling and as modeling needs require.
I will work with Dr. Knapp to develop, implement, and document agent-based models for each aim, including calibration and validation using empirical data, with emphasis on the agent-based evolving network models of Aim 3.
I will work with Dr. Prosperi to design, implement, and document the transfer learning procedure of Aim 2.2.
In each case, I will also contribute to the open-access publication of these models on established code repositories.
Finally, I will work with Dr. Sego and the Model and Data Sharing Core to convert and validate models in multiple software languages and on diverse platforms, and to publish them with clear documentation including fully worked through examples online.

\bigskip

\noindent
Ongoing and recently completed projects that I would like to highlight include:

\bigskip

\grantinfo{The University of Florida Clinical and Translational Sciences Institute Pilot Award (Parent award: NIH/NCATS UL1 TR001427)}
{Role: PI}
{08/01/22--04/30/24}
{Efficient Modeling of Individualized COVID-19 Mortality Risk}

\bigskip

\grantinfo{R Consortium Infrastructure Steering Committee Grant Program. (24-ISC-1-06)}
{Role: PI}
{10/01/24--05/31/25}
{Modular, interoperable, and extensible topological data analysis in R}

\bigskip

\noindent
Citations:

\begin{enumerate}

%\item \strong{J.C. Brunson} (2015), Triadic analysis of affiliation networks, \emph{Network Science} 3(4): 480--508.
%\item \strong{J.C. Brunson}, X. Wang, and R.C. Laubenbacher (2017). Effects of research complexity and competition on the incidence and growth of coauthorship in biomedicine, \emph{PLoS One} 12(3): e0173444.
%\item \strong{J.C. Brunson} and R.C. Laubenbacher (2018). Applications of network analysis to routinely collected healthcare data: a systematic review, \emph{Journal of the American Medical Informatics Association} 25(2): 210--221.
%\item \strong{J.C. Brunson}, T.P. Agresta, and R.C. Laubenbacher (2019). Sensitivity of comorbidity network analysis, \emph{JAMIA Open} 3(1): 94--103.
\item M. Terasaki, \strong{J.C. Brunson}, and J. Sardi (2020). Analysis of the three dimensional structure of the kidney glomerulus capillary network, \emph{Scientific Reports} 10, 20334.
\item \strong{J.C. Brunson} (2020). ggalluvial: Layered Grammar for Alluvial Plots, \emph{Journal of Open Source Software} 5(49): 2017.
%\item \strong{J.C. Brunson}, Y. Skaf (2022). Fixed and adaptive landmark sets for finite pseudometric spaces. arXiv 2212.09826 [Preprint]. 2023 Jan 13 (v2).  Available from: \url{https://arxiv.org/abs/2212.09826}.
\item L. Riley, \strong{J.C. Brunson}, S Eydgahi, M. Brantly, J. Lascano (2023). Development of a Risk Score to Increase Detection of Severe Alpha-1 Antitrypsin Deficiency, \emph{ERJ Open Research} 9(5): 00302.
\item A.D. Guastello, \strong{J.C. Brunson}, N. Sambuco, L.P. Dale, N.A. Tracy, B.R. Allen, C.A. Mathews (2024). Predictors of professional burnout and fulfillment in a longitudinal analysis on nurses and healthcare workers in the COVID-19 pandemic, \emph{Journal of Clinical Nursing} 33(1): 288--303.

\end{enumerate}

\end{statement}


%------------------------------------------------------------------------------
\section{Positions, Scientific Appointments, and Honors}

%\subsection*{Positions}
\begin{datetbl}
2020--Present     & Research Assistant Professor, Laboratory for Systems Medicine, University of Florida, Gainesville, FL  \\
2017--2020 & Postdoctoral Fellow, Skeletal, Craniofacial \& Oral Biology Training Program, UConn Health, Farmington, CT \\
2014--2017 & Postdoctoral Fellow, Center for Quantitative Medicine, UConn Health, Farmington, CT \\
2014       & Adjunct Professor, Department of Mathematics, Radford University, Radford, VA \\
2010--2013 & Research Assistant, Virginia Bioinformatics Institute, Virginia Tech, Blacksburg, VA \\
\end{datetbl}

%\subsection*{Honors}
%\begin{datetbl}
%2017       & Union Representative of the Year, UHP \\
%\end{datetbl}


%------------------------------------------------------------------------------
\section{Contribution to Science}

\begin{enumerate}

\item \emph{Network science.}
As a Research Assistant and mentor, together with a team of undergraduates, I was introduced to network analysis and scientometrics at a summer program organized around systems biology.
Our study of the mathematics literature, and a later collaboration focused on biomedicine, combined conventional and original tools to describe global changes in collaboration patterns in both communities over recent decades, and our results have been cited in both follow-up research and several commentaries on scientific practice.
I have since kept up an active program of network science, which has involved collaborations with domain experts in immune response and cell biology on original applications of advanced graph theory to modeling cell signaling and capillary development.
Network science has also been a major part of my more recent work involving administrative healthcare data, as detailed in the next section.

\begin{enumerate}
\item \strong{J.C. Brunson}, S. Fassino, A. McInnes, M. Narayan, B. Richardson, C. Frank, P. Ion, and R.C. Laubenbacher (2014). Evolutionary events in a mathematical sciences research collaboration network, \emph{Scientometrics} 99(3): 973--998.
\item \strong{J.C. Brunson} (2015), Triadic analysis of affiliation networks, \emph{Network Science} 3(4): 480--508.
%\item \strong{J.C. Brunson}, X. Wang, and R.C. Laubenbacher (2017). Effects of research complexity and competition on the incidence and growth of coauthorship in biomedicine, \emph{PLoS One} 12(3): e0173444.
\item \strong{J.C. Brunson}, T.P. Agresta, and R.C. Laubenbacher (2020). Sensitivity of comorbidity network analysis, \emph{JAMIA Open} 3(1): 94--103.
\item M. Terasaki, \strong{J.C. Brunson}, and J. Sardi (2020). Analysis of the three dimensional structure of the kidney glomerulus capillary network, \emph{Scientific Reports} 10, 20334.
\end{enumerate}

\item \emph{Health informatics.}
I self-trained at CQM as a data scientist, which entails a combination of statistical literacy, computational programming, and consultancy, with a focus on the pre-processing, analysis, and modeling of administrative healthcare data sets.
I conducted a systematic review of studies that use network analysis to study healthcare data, a literature that contains myriad projects in diverse domains. My review synthesized this literature and taxonomized its methods in service to the field, but it also revealed to me that the increasingly popular use of ``comorbidity networks'' suffers from inconsistent methodology and uncertain validity. In response, and with a physician--informaticist colleague, we conducted a sensitivity analysis of techniques that have been used to study comorbidity networks, which we hope will encourage more consistent and transparent work in future.
In addition to my scientific work, I completed a training collaboration with the Office of the State Comptroller of Connecticut, analyzing prescription patterns in claims data as part of an investigation into compound pharmacy fraud, and I collaborated with a pharmacist--scientist to develop a set of software packages to expedite comparative effectiveness research using claims databases.
I use my background in reproducibility to ensure in each case that my analysis is reproducible, auditable, and reusable. I have also worked to bring somewhat niche geometric and topological data analysis techniques into more mainstream use by way of open source software development.

\begin{enumerate}
\item \strong{J.C. Brunson} and R.C. Laubenbacher (2018). Applications of network analysis to routinely collected healthcare data: a systematic review, \emph{Journal of the American Medical Informatics Association} 25(2): 210--221.
%\item \strong{J.C. Brunson}, T.P. Agresta, and R.C. Laubenbacher (2020). Sensitivity of comorbidity network analysis, \emph{JAMIA Open} 3(1): 94--103.
\item \strong{J.C. Brunson} (2020). ggalluvial: Layered Grammar for Alluvial Plots, \emph{Journal of Open Source Software} 5(49): 2017.
%\item \strong{J.C. Brunson} and E. Paul (2022). ordr: A `tidyverse' Extension for Ordinations and Biplots, R package version 0.1.1. \nolinkurl{https://cran.r-project.org/package=ordr}
\item A. Cohen, P. McCall-Junkin, and \strong{J.C. Brunson} (2024). Clinical prediction with localized modeling using similarity-based cohorts: A scoping review. medRxiv 2024.06.04.24308433 [Preprint]. 2024 Jun 4.  Available from: \url{https://doi.org/10.1101/2024.06.04.24308433}.
\item E. Nitch-Griffin, A. Peterson, Y. Skaf, and \strong{J.C. Brunson} (2024). Reproducibility and Robustness of Localized Mortality Prediction. medRxiv 2024.06.04.24308417 [Preprint]. 2024 Jun 4.  Available from: \url{https://doi.org/10.1101/2024.06.04.24308417}.
\end{enumerate}

\item \emph{Descriptive and predictive modeling.}
Since joining the University of Florida (UF), I have begun collaborations with several clinician--researchers, in Psychiatry, Psychology, and Family Medicine as well as in Pulmonary.
My work includes some conventional statistical analysis, for example to describe determinants of healthcare worker burnout during the pandemic or to predict post-transplant outcomes among lung recipients, as well as cutting-edge model development, for example integer programming to develop a simple risk score for severe AATD that we hope will inform screening recommendations.
As I came to appreciate the limitations of network analysis to predictive modeling and hypothesis testing, I developed an interest topological data analysis (TDA). This led an MD--PhD student and I to develop sampling procedures to improve TDA selection procedures for predictive models using health data, which violates many of the implicit assumptions satisfied by so-called ``point cloud'' data.
This led me to couple an artificial intelligence approach called case-based reasoning into a machine learning framework in order to improve the predictive accuracy of models without cost to their interpretability. We are using these ``localized'' models to enhance mortality prediction in ICU and cardiac outcomes for COVID-19, using patient data obtained from electronic health records. A scoping review of this approach and two empirical studies, all with student collaborators, are submitted or in preparation, and this approach is the basis for my CTSI Precision Health Initiative pilot award.

\begin{enumerate}
%\item \strong{J.C. Brunson}, X. Wang, and R.C. Laubenbacher (2017). Effects of research complexity and competition on the incidence and growth of coauthorship in biomedicine, \emph{PLoS One} 12(3): e0173444.
\item \strong{J.C. Brunson}, Y. Skaf (2022). Fixed and adaptive landmark sets for finite pseudometric spaces. arXiv 2212.09826 [Preprint]. 2023 Jan 13 (v2).  Available from: \url{https://arxiv.org/abs/2212.09826}.
\item L. Riley, \strong{J.C. Brunson}, S Eydgahi, M. Brantly, J. Lascano (2023). Development of a Risk Score to Increase Detection of Severe Alpha-1 Antitrypsin Deficiency, \emph{ERJ Open Research} 9(5): 00302.
\item A.D. Guastello, \strong{J.C. Brunson}, N. Sambuco, L.P. Dale, N.A. Tracy, B.R. Allen, C.A. Mathews (2024). Predictors of professional burnout and fulfillment in a longitudinal analysis on nurses and healthcare workers in the COVID-19 pandemic, \emph{Journal of Clinical Nursing} 33(1): 288--303.
\item D. Nord*, \strong{J.C. Brunson}*, L. Langerude, H. Moussa, B. Gill, T. Machuca, M. Rackauskas, A.K. Sharma, C. Lin, A. Emtiazjoo, C. Atkinson (2024). Predicting Primary Graft Dysfunction in Lung Transplantation: Machine Learning--Guided Biomarker Discovery. bioRxiv 2024.05.24.595368 [Preprint]. 2024 May 30. Available from: \url{https://doi.org/10.1101/2024.05.24.595368}.
%\item \strong{J.C. Brunson}, J. Jaber, S. Nandavaram, A. Emtiazjoo, D.C. Patel, D. Gomez Manjarres (2024). Racial-ethnic disparities in post-transplant outcomes of lung recipients.
\end{enumerate}

%\item \emph{Topological data analysis.}
%My scientometric work raised a remarkably under-studied question: How can triads of actors be understood in an affiliation network context? A satisfactory answer required the machinery of homological algebra, which had been essential to my dissertation project but which had not been adapted to this setting. As I came to appreciate the limitations of network analysis to exploratory analysis, predictive modeling, and hypothesis testing with healthcare data, this success was an important factor in my decision to pursue topological data analysis (TDA). My TDA work, joint with mathematical and clinical collaborators, is focused on two entangled research programs. One is to couple topological methods with complex predictive models in order to identify health-related features and patient phenotypes that help explain the predictions and the performances of these models. This is joint work with mathematical and clinical collaborators, including MD--PhD student Yara Skaf, who has been instrumental to its design and conduct since she joined our lab and will co-author our first manuscript. The other is to synthesize existing and original TDA tools into a collection of software packages integrated into widely-used data science conventions, in order to make rigorous and efficient topological methods accessible to data scientists. This is joint work with student researchers at several institutions, including Ms. Skaf, and serves an important role in allowing student researchers to participate directly in methods development and experimentation. My own primary project is a topological approach to the problem of the unequal performance of predictive models in healthcare, which connects to work in health disparities, algorithmic bias, and algorithmic stewardship.

%\begin{enumerate}
%\item \strong{J.C. Brunson}, Y. Skaf (2022). Fixed and adaptive landmark sets for finite pseudometric spaces. arXiv 2212.09826 [Preprint]. 2023 Jan 13 (v2).  Available from: \url{https://arxiv.org/abs/2212.09826}.
%\item R.R. Wadhwa, M. Piekenbrock, and \strong{J.C. Brunson}. tdaverse: An R package collection for topological data analysis. \nolinkurl{http://systemsmedicine.pulmonary.medicine.ufl.edu/software/r-package-ecosystem-tda/}
%\end{enumerate}

%More recently, I mentored a team of undergraduates in developing risk models for myocardial infarction (MI) using the open-access critical care database MIMIC-III. We built analysis pipelines to contribute to the MIMIC-III code repository and built several MI triage and risk models, which are being written up for submission to a medical journal.

\end{enumerate}


\subsubsection*{Complete List of Published Work in MyBibliography:}
\url{https://www.ncbi.nlm.nih.gov/myncbi/browse/collection/47860258/?sort=date&direction=ascending}

%\subsubsection*{Complete List of Publications at ORCiD:}
%\url{https://orcid.org/0000-0003-3126-9494}

%\subsubsection*{Complete List of Reviews at Publons:}
%\url{https://publons.com/researcher/1676577/jason-cory-brunson/peer-review/}


\end{document}

